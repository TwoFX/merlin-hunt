\documentclass[fontsize=10pt,a4paper,DIV=12,parskip=half]{scrarticle}

\usepackage[utf8]{inputenc}
\usepackage[T1]{fontenc}
\usepackage{lmodern}
\usepackage{csquotes}
\usepackage{hyperref}
\usepackage{booktabs}

\title{Merlin Hunt}
\subtitle{General information and advice}
\date{Last updated: \today}

\begin{document}
\maketitle

The canonical location of this document is \url{https://static.markushimmel.de/information.pdf}.

\subsection*{General information and schedule}

The Merlin Hunt is a virtual puzzle hunt for people who know how to program.
It is written and organized by Markus Himmel. While it is not being advertised
publically, participation is open to everyone.

\paragraph{Communication} The main medium of communication is Discord. Every participant should join the following
Discord server:

\begin{center}
	\url{https://discord.gg/ncmky4ygDc}
\end{center}

\paragraph{Schedule}
The date of the contest is February 25, 2023. The schedule is as follows (all times in CET):

\begin{center}
	\begin{tabular}{@{}ll@{}}
		\toprule
		Time & Event \\ \midrule
		09:45 & Last info meeting on Discord \\
		10:00 & Start of contest \\
		00:00 & End of contest \\
		00:05 & Award ceremony on Discord \\ \bottomrule
	\end{tabular}
\end{center}

\paragraph{Contest location}
The puzzle hunt can be completely played online using a computer. All official
communication will be via Discord.

There will be a limited amount of space for teams to participate onsite in Karlsruhe (just for the fun of it).
If you would like to participate onsite, please indicate this during sign-up. If
there are more interested teams than there is space in my flat, I will decide which
teams are allowed to come.

If you participate onsite, please bring a pair of headphones!

\paragraph{Teams and registration}

Participants should participate in teams of two or three people. Forming larger
teams is possible but not recommended. Participating alone is also possible but
strongly discouraged.

To register a team, send me\footnote{Discord handle: TwoFx\#6343} a private
message on Discord.

Please include the following information:
\begin{itemize}
	\item Your team name
	\item The names and Discord handles of all team members
	\item Which team member is the captain
	\item Whether or not you would like to participate onsite in Karlsruhe
\end{itemize}

While it will be possible to sign up right until the contest starts, please
sign up as early as possible.

If you do not have a team, feel free to message me and I will try to match you up
with other individuals who are looking for a team.

\subsection*{Competition format}
The puzzle hunt is structured as an ordered series of tasks. A task is a
zip file containing the following things:
\begin{itemize}
	\item The \textit{flag} of the previous task: a file with the extension
		\texttt{.flag} that you should upload to DOMJudge (see below) to obtain
		points for the previous problem.
	\item One or more files that make up the puzzle that you have to solve for this
		task.
	\item The next task, as an encrypted zip file with extension \texttt{.zip.gpg}.
		The solution of the task is the passphrase needed to decrypt the next task.
\end{itemize}

The initial zip file containing the first task will be distributed via Discord
shortly before the contest. The password to decrypt the task will be
distributed via Discord as soon as the contest starts. You should download the
initial zip file in advance as it will be quite large.

\paragraph{DOMJudge}
Each team's progress is tracked via an online system called DOMJudge. The URL of
the DOMJudge system is
\begin{center}
	\url{https://domjudge.markushimmel.de/}.
\end{center}

As soon as you successfully decrypt a zip file, you should upload the flag
contained in the archive to DOMJudge in order to obtain points for the task you
just solved.

You will obtain credentials for DOMJudge after signing up. The same credentials
will work for the practice session and the main contest.

On the DOMJudge page, there will be a live ranking of all participating teams.
The teams will be sorted using ICPC scoring rules, which basically means that
to appear at the top of the leaderboard, you should solve many puzzles and do
so quickly and with few hints.

\paragraph{Example}
Here is an example of how the process of participating in the puzzle hunt looks.
You will be able to practice the process during the practice session; see below.

\begin{itemize}
	\item Before the contest, you download the first task. It is a zip file called
		\texttt{gettingstarted.zip}. Inside the zip archive, there are two files:
		\texttt{README.md} and \texttt{thefirsttask.zip.gpg}. The readme file tells
		you that you will receive the password to decrypt the encrypted file once
		the contest has started.
	\item As soon as the contest starts, the password for \texttt{thefirsttask.zip.gpg}
		is announced on Discord. You decrypt the archive and find three files inside:
		\texttt{gettingstarted.flag}, \texttt{puzzle.txt} and \texttt{thesecondtask.zip.gpg}.
	\item You upload \texttt{gettingstarted.flag} to DOMJudge and receive one point.
	\item You solve the puzzle contained in \texttt{puzzle.txt}. The solution is the
		sentence \enquote{Elephants are larger than pixels}.
	\item You use this phrase to decrypt \texttt{thesecondtask.zip.gpg} and find
		four files inside the archive: \texttt{thefirsttask.flag}, \texttt{secondpuzzle.txt},
		\texttt{secondpuzzle.jpg} and \texttt{thethirdtask.zip.gpg}.
	\item You upload \texttt{thefirsttask.flag} to DOMJudge and receive one point.
	\item You solve the puzzle contained in \texttt{secondpuzzle.*}, use this to decrypt the
		next problem archive, upload the flag, solve the next puzzle, and so on.
\end{itemize}

\paragraph{Hints}
If you get stuck on a puzzle, it is possible to request a hint. You may only
request a hint if you have been working on a puzzle for at least one hour
(as measured by the last successful submission on DOMJudge).

To request a hint, you should do two things:
\begin{enumerate}
	\item In DOMJudge, submit the \texttt{.flag} file of a previous puzzle as an attempt for the puzzle you
		want to request a hint for. You will receive a verdict of Wrong Answer. If you solve
		the puzzle later, you will incur a 20 minute time penalty for the hint.
	\item Send me a private message on Discord.
\end{enumerate}

I will then join you for a quick voice call to get you un-stuck.

\subsection*{Rules and advice}
\begin{itemize}
	\item Before the contest, you should make sure that you are able to decrypt
		files that were encrypted using GnuPG. On Linux, this is accomplished
		using the command-line tool \texttt{gpg}. On Debian, Ubuntu, Mint and derivatives, you can
		install it by saying
		\begin{center}
			\texttt{sudo apt install gnupg}
		\end{center}
		You can decrypt a file using the invocation
		\begin{center}
			\texttt{gpg -d task.zip.gpg > task.zip}
		\end{center}
		There are also GUI tools available on Windows and Linux.
	\item It is strongly recommended that at least one team member has access
		to a reasonably powerful Linux machine or VM.
	\item Using multiple computers per team and using the internet is permitted.
	\item After solving a puzzle, you should know exactly what the solution to
		the puzzle is. Do not try to brute force the passphrase, even if you
		believe that you have reduced the number of possible passphrases to a
		small number.
	\item Some puzzles will require interacting with externally hosted services.
		Please be considerate and do not spam them with a large number of automated
		requests.
	\item The name of a puzzle is part of the puzzle. The full name of each puzzle
		can be viewed in the DOMJudge submit dialog.
\end{itemize}

\subsection*{Practice session}
There will be a practice session to test your understanding of the contest format and
get familiar with \texttt{gpg}, DOMJudge and so on. It will work just like the real
contest, except that the problems are uninteresting dummy problems and you will have
multiple days to participate at your leisure.

Participation in the practice session is optional but recommended.
You can spend as much or as little time as you like
on the practice session, but an hour should be more than enough.

The practice session will run
from February 10 to February 24. The problem package and initial passphrase for the practice session
will be made available via Discord.

\subsection*{Extended contest}
After the contest has ended, the contest infrastructure will remain accessible for
a few more weeks or months. There will be an extended contest in DOMJudge so that you can
continue solving the puzzles past the end of the contest if you wish.

\end{document}
