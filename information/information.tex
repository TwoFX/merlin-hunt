\documentclass[fontsize=10pt,a4paper,DIV=12]{scrarticle}

\usepackage[utf8]{inputenc}
\usepackage[T1]{fontenc}
\usepackage{lmodern}
\usepackage{csquotes}
\usepackage{hyperref}
\usepackage{booktabs}

\title{Merlin Hunt}
\subtitle{General information and advice}
\date{}

\begin{document}
\maketitle

\section{General information and schedule}

The main medium of communication is Discord. Every participant must join the following
Discord server:

\begin{center}
	TBD
\end{center}

\subsection{Schedule}
The date of the contest is TBD. The schedule is as follows (all times in CET):

\begin{center}
	\begin{tabular}{@{}ll@{}}
		\toprule
		Time & Event \\ \midrule
		09:00 & Start of late signup \\
		09:45 & Last info meeting on Discord \\
		10:00 & Start of contest \\
		00:00 & End of contest \\
		00:05 & Award ceremony on Discord \\ \bottomrule
	\end{tabular}
\end{center}

\subsection{Teams and registration}

Participants should participate in teams of two or three people. Forming larger
teams is possible but not recommended. Participating alone is also possible but
strongly discouraged.

To register a team, send me a private message on Discord. Please include the team
name as well as the names and Discord handles of all team members.

If you do not have a team, feel free to message me and I will try to match you up
with other individuals who do not have a team.

\section{Competition format}
The puzzle hunt is structured as an ordered series of tasks. A task is a
zip file containing the following things:
\begin{itemize}
	\item The \textit{flag} of the previous task: a file with the extension
		\texttt{.flag} that you should upload to DOMJudge (see below) to obtain
		points for the previous problem.
	\item One or more files that make up the puzzle that you have to solve for this
		task.
	\item The next task, as an encrypted zip file with extension \texttt{.zip.gpg}.
		The solution of the task is the passphrase needed to decrypt the next task.
\end{itemize}

\subsection{DOMJudge}
Each team's progress is tracked via an online system called DOMJudge. The URL of
the DOMJudge system is
\begin{center}
	\url{https://domjudge.markushimmel.de/}.
\end{center}

As soon as you successfully decrypt a zip file, you should upload the flag
contained in the archive to DOMJudge in order to obtain points for the task you
just solved.

You will obtain credentials for DOMJudge after signing up. The same credentials
will work for the practice session and the main contest.

\subsection{Example}
Here is an example of how the process of participating in the puzzle hunt looks.
You will be able to practice the process during the practice session; see below.

\begin{itemize}
	\item Before the contest, you download the first task. It is a zip file called
		\texttt{gettingstarted.zip}. Inside the zip archive, there are two files:
		\texttt{README.md} and \texttt{thefirsttask.zip.gpg}. The readme file tells
		you that you will receive the password to decrypt the encrypted file once
		the contest has started.
	\item As soon as the contest starts, the password for \texttt{thefirsttask.zip.gpg}
		is announced on Discord. You decrypt the archive and find three files inside:
		\texttt{gettingstarted.flag}, \texttt{puzzle.txt} and \texttt{thesecondtask.zip.gpg}.
	\item You upload \texttt{gettingstarted.flag} to DOMJudge and receive one point.
	\item You solve the puzzle contained in \texttt{puzzle.txt}. The solution is the
		sentence "Elephants are larger than pixels".
	\item You use this phrase to decrypt \texttt{thesecondtask.zip.gpg} and find
		four files inside the archive: \texttt{thefirsttask.flag}, \texttt{secondpuzzle.txt},
		\texttt{secondpuzzle.jpg} and \texttt{thethirdtask.zip.gpg}.
	\item You upload \texttt{thefirsttask.flag} to DOMJudge and receive one point.
	\item You solve the puzzle contained in \texttt{secondpuzzle.*}, use this to decrypt the
		next problem archive, upload the flag, solve the next puzzle, and so on.
\end{itemize}

\section{Rules and hints}
\begin{itemize}
	\item Before the contest, you should make sure that you are able to decrypt
		files that were encrypted using GnuPG. On Linux, this is accomplished
		using the command-line tool \texttt{gpg}. On Debian/Ubuntu/Mint, you can
		install it by saying
		\begin{center}
			\texttt{sudo apt install gnupg}
		\end{center}
		You can decrypt a file using the invocation
		\begin{center}
			\texttt{gpg -d task.zip.gpg > task.zip}
		\end{center}
		There are also GUI tools available on Windows and Linux.
	\item It is strongly recommended that at least one team member has access
		to a reasonably powerful Linux machine or VM.
	\item After solving a puzzle, you should know exactly what the solution to
		the puzzle is. Do not try to brute force the passphrase, even if you
		believe that you have reduced the number of possible passphrases to a
		small number.
	\item Some puzzles will require interacting with externally hosted services.
		Please be considerate and do not spam them with a large number of autmated
		requests.
\end{itemize}

\section{Practice session}
There will be a practice session to test your understanding of the contest format and
get familiar with \texttt{gpg}, DOMJudge and so on. The practice session will run
from TBD to TBD. The problem package and initial passphrase for the practice session
will be made available via Discord.

\end{document}
